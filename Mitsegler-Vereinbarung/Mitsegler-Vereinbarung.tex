\documentclass[a4paper,12pt]{article}
\usepackage[T1]{fontenc}
\usepackage[utf8]{inputenc}
\usepackage{lmodern}
\usepackage{ngerman}

\usepackage{booktabs}
\usepackage{avant}
\usepackage{mathptmx}
\usepackage{titlesec}
\usepackage{eurosym}

\usepackage{geometry}
\setlength{\parindent}{0mm}
\setlength{\parskip}{2.5mm}
\setlength{\textwidth}{16cm}
\setlength{\textheight}{24.5cm}

\setlength{\oddsidemargin}{0cm}
\setlength{\topmargin}{-1.25cm}


% Paragraphen-Titel
\titleformat{\section}{\sffamily\large\bfseries}{\thesection}{16pt}{\large}
\titlespacing*{\section}{0mm}{7mm}{0mm}

\titleformat{\subsection}{\sffamily\normalsize\bfseries}{\thesubsection}{16pt}{\normalsize}
\titlespacing*{\subsection}{0mm}{5mm}{0mm}

% Leerstellen
\newcommand{\openlength}[1]{\hspace{0.25cm} \makebox[#1]{\hrulefill}\hspace{0.25cm}}
\newcommand{\open}{\openlength{4cm}}

% Personen
\newcommand{\skipper}{\underline{Skipper}} % hier Namen der:des Skipper:in eintragen
\newcommand{\vercharterer}{\open} % hier Namen des Charterunternehmens eintragen

\begin{document}

\section*{Mitsegler:innen-Vereinbarung}

für den Segeltörn vom \open\ bis \open \\
in \open\ auf der Jacht \open \\
von \vercharterer, mit Ausgangshafen \open\ 
bei dem die unter Ziffer~\ref{sec:Mitsegler} aufgeführten Personen Mitsegler:innen sind.


\section{Chartervertrag}
\label{sec:Chartervertrag}

Der zwischen \skipper\ und dem Vercharterer für die Crew geschlossene Chartervertrag vom ist Grundlage dieser Vereinbarung.
Jede:r Mitsegler:in kann auf Wunsch eine Kopie dieses Chartervertrages erhalten und ist mit den insoweit zugrunde gelegten Regelungen einverstanden.
\skipper\ ist nicht Veranstalter im Sinne des Reisevertragsrechts.


\section{Kosten}
\label{sec:Kosten}

Die Mitsegler:innen tragen sämtliche Törnkosten gemeinsam zu gleichen Teilen.
Dies sind insbesondere die \textit{Charterkosten} und die \textit{Bordkasse} (u.a. Verpflegung, Treibstoff, Hafengelder, Gebühren etc.).
Darüber hinaus tragen die Mitsegler:innen auch Kosten gemeinsam, die sich aus der \textit{Nichterfüllung} des Chartervertrages ergeben können und etwaige Kosten im \textit{Schadensfall}, soweit dafür keine Versicherung eintritt und ein Schaden nicht vorsätzlich von einer:m Mitsegler:in verursacht wurde.


Die Charterkosten betragen \mbox{\openlength{1cm}\euro}.\\
Darin enthalten sind:
\begin{itemize}
	\item Chartergebühr
	\item Vermittlungsgebühr
	\item Endreinigung
	\item Versicherung
	\item Skipper-Haftpflichtversicherung
\end{itemize}

Die Selbstbeteiligung für die Kaskoversicherung beträgt \mbox{\openlength{1cm}\euro}.
Dies entspricht der vor Ort zu hinterlegenden Kaution in bar/Scheck/Kreditkarte.
Bzgl. der Selbstbeteiligung im Schadensfall ist eine Skipper-Haftpflichtversicherung abgeschlossen worden.

Die Anreise (z.B. Flug) wird von jeder:m Mitsegler:in individuell auf eigene Kosten organisiert.

Jede:r Mitsegler:in verpflichtet sich, die auf sie:ihn entfallende erste Rate in Höhe von\\
\mbox{\openlength{1cm}\euro} bis zum \open\ an \skipper\ zu entrichten.
Die weiteren Charterkosten sind spätestens zum Törnantritt an \skipper\ zu entrichten.

Bei Reiserücktritt eines Mitseglers, gleich aus welchem Grund, zahlt dieser seinen Anteil an den Charterkosten, soweit dafür nicht eine Reise-Rücktrittskosten-Versicherung eintritt oder die übrigen Mitsegler:innen darauf ausdrücklich verzichten.


\section{Schiffsführer}
\label{sec:Skipper}

Verantwortlicher Schiffsführer ist \skipper,\\
stellvertretender Schiffsführer ist \open.

Der Schiffsführer versichert, dass er die notwendigen Erfahrungen, Kenntnisse und Qualifikationen besitzt, um die Jacht unter Segel und Motor sicher zu führen.
Er weist die Mitsegler in die Bedienung der Jacht ein und führt eine gründliche Sicherheitseinweisung durch.


\section{Pflichten der Mitsegler:innen}
\label{sec:Pflichten}

Jede:r Mitsegler:in beachtet die Anweisungen des Schiffsführers und informiert ihn beziehungsweise die:den jeweiligen Wachführer:in.
Jede:r Mitsegler:in achtet selbst auf die persönliche Sicherheit und trägt bei Bedarf und in jedem Falle auf Anweisung des Schiffsführers eine angemessene persönliche Schutzausstattung, insbesondere Rettungsweste und Lifebelt.
Alle Mitsegler:innen verhalten sich stets so, dass das Risiko von Körper- und Sachschäden minimiert wird.

Alle Mitsegler:innen erklären, dass sie die persönlichen Voraussetzungen (insb. Gesundheit und Kondition) für die Törnteilnahme erfüllen.


\section{Haftungsausschluß}
\label{sec:Haftung}

Durch den Betrieb einer Jacht können Mitsegler:innen Körper- und Sachschäden erleiden.
Dieses Risiko nimmt jede:r Mitsegler:in in Kauf und nimmt \textit{auf eigene Gefahr} an dem Törn teil.

Alle Mitsegler:innen verzichten auf Ersatzansprüche aus allen rechtlichen Gesichtspunkten für Personen- und Sachschäden gegen den Schiffsführer, die anderen Mitsegler und den Eigner, sofern dieser Mitsegler ist, wenn der Schaden auf fahrlässigem Verhalten beruht.
Der Haftungsausschluss gilt nicht, soweit Schäden vorsätzlich verursacht wurden oder von einer Haftpflichtversicherung getragen werden.
Für entstandene Schäden haftet nicht der Schiffsführer, sondern die gesamte Crew.


\section{Gültigkeit}
\label{sec:Gueltigkeit}

Sollten Teile dieser Vereinbarung ungültig oder undurchführbar sein oder werden, soll dies die Wirksamkeit der anderen Teile dieser Vereinbarung nicht beeinträchtigen.
Gleiches gilt, wenn sich herausstellt, dass die Vereinbarung eine Regelungslücke enthält.
Anstelle des unwirksamen/ undurchführbaren Teils oder zur Ausfüllung der Lücke soll diese Vereinbarung so ausgelegt werden, dass sie dem beabsichtigten Zweck möglichst nahe kommt.

Streitigkeiten beurteilen sich nach deutschem Recht.
Mündliche Nebenabreden sind nicht getroffen.
Änderungen und/oder Ergänzungen dieser Vereinbarung bedürfen der Schriftform.
Dies gilt insbesondere auch für eine Aufhebung des Schriftformerfordernisses.
Jede:r Mitsegler:in bestätigt, eine Ausfertigung dieses Vertrages für seine Unterlagen erhalten zu haben.


\section{Mitsegler:innen}
\label{sec:Mitsegler}

\subsection*{Liste der Mitsegler:innen}

\begin{center}
\begin{tabular}{@{}rlllll@{}}
	\toprule
	   & Name          & Geburtsdatum & Segelschein? & Telefon       & E-Mail        \\ \midrule
	1. & \hspace*{3cm} &              &              & \hspace*{3cm} & \hspace*{3cm} \\ [9pt] \hline
	2. &               &              &              &               &               \\ [9pt] \hline
	3. &               &              &              &               &               \\ [9pt] \hline
	4. &               &              &              &               &               \\ [9pt] \hline
	5. &               &              &              &               &               \\ [9pt] \hline
	6. &               &              &              &               &               \\ [9pt] \hline
	7. &               &              &              &               &               \\ [9pt] \hline
	8. &               &              &              &               &               \\ [9pt] \bottomrule
\end{tabular}
\end{center}


\subsection*{Unterschriften}

\subsubsection*{1.}
\begin{tabular}{rp{12pt}l}
Ort, Datum:   && \openlength{4cm}  \\ \\[9pt]
Unterschrift: && \openlength{8cm}  \\ \\[3pt]
\end{tabular}

\subsubsection*{2.}
\begin{tabular}{rp{12pt}l}
Ort, Datum:   && \openlength{4cm}  \\ \\[9pt]
Unterschrift: && \openlength{8cm}  \\ \\[3pt]
\end{tabular}

\subsubsection*{3.}
\begin{tabular}{rp{12pt}l}
Ort, Datum:   && \openlength{4cm}  \\ \\[9pt]
Unterschrift: && \openlength{8cm}  \\ \\[3pt]
\end{tabular}

\subsubsection*{4.}
\begin{tabular}{rp{12pt}l}
Ort, Datum:   && \openlength{4cm}  \\ \\[9pt]
Unterschrift: && \openlength{8cm}  \\ \\[3pt]
\end{tabular}

\subsubsection*{5.}
\begin{tabular}{rp{12pt}l}
Ort, Datum:   && \openlength{4cm}  \\ \\[9pt]
Unterschrift: && \openlength{8cm}  \\ \\[3pt]
\end{tabular}

\subsubsection*{6.}
\begin{tabular}{rp{12pt}l}
Ort, Datum:   && \openlength{4cm}  \\ \\[9pt]
Unterschrift: && \openlength{8cm}  \\ \\[3pt]
\end{tabular}

\subsubsection*{7.}
\begin{tabular}{rp{12pt}l}
Ort, Datum:   && \openlength{4cm}  \\ \\[9pt]
Unterschrift: && \openlength{8cm}  \\ \\[3pt]
\end{tabular}

\subsubsection*{8.}
\begin{tabular}{rp{12pt}l}
Ort, Datum:   && \openlength{4cm}  \\ \\[9pt]
Unterschrift: && \openlength{8cm}  \\ \\[3pt]
\end{tabular}

\end{document}
